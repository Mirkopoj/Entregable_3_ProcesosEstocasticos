\documentclass[12pt,letterpaper]{article}     % Tipo de documento y otras especificaciones
\usepackage[utf8]{inputenc}                   % Para escribir tildes y eñes
\usepackage[spanish]{babel}                   % Para que los títulos de figuras, tablas y otros estén en español
\addto\captionsspanish{\renewcommand{\tablename}{Tabla}}					% Cambiar nombre a tablas
\addto\captionsspanish{\renewcommand{\listtablename}{Índice de tablas}}		% Cambiar nombre a lista de tablas
\usepackage{geometry}                         
\geometry{left=18mm,right=18mm,top=21mm,bottom=21mm} % Tamaño del área de escritura de la página
\usepackage{ucs}
\usepackage{amsmath}      % Los paquetes ams son desarrollados por la American Mathematical Society
\usepackage{amsfonts}     % y mejoran la escritura de fórmulas y símbolos matemáticos.
\usepackage{amssymb}
\usepackage{graphicx}     % Para insertar gráficas
\usepackage[lofdepth,lotdepth]{subfig}	% Para colocar varias figuras
\usepackage{unitsdef}	  % Para la presentación correcta de unidades
\usepackage{pdfpages}   %incluir paginas de pdf externo, para los anexos
\usepackage{appendix}   %para los anexos
\renewcommand{\unitvaluesep}{\hspace*{4pt}}	% Redimensionamiento del espacio entre magnitud y unidad
\usepackage[colorlinks=true,urlcolor=blue,linkcolor=black,citecolor=black]{hyperref}     % Para insertar hipervínculos y marcadores
\usepackage{float}		% Para ubicar las tablas y figuras justo después del texto
\usepackage{booktabs}	% Para hacer tablas más estilizadas
\usepackage{color}
\batchmode
%\bibliographystyle{plain} 
\pagestyle{plain} 
\pagenumbering{arabic}
\usepackage{lastpage}
\usepackage{fancyhdr}	% Para manejar los encabezados y pies de página
\usepackage{mdframed}
\pagestyle{fancy}		% Contenido de los encabezados y pies de pagina
\setlength{\headheight}{15pt}
\usepackage{multicol}   % Para varias columnas
\usepackage[export]{adjustbox}
\usepackage{ragged2e}
%\usepackage{pgfplots}
%%%%
%---------------------------Definición del environment resumen---------------------------
\newcounter{resumen}
\setcounter{resumen}{0}
\def\theejemplo{\thechapter.\arabic{resumen}}

\newenvironment{resumen}
{	
	\begin{center}
	\begin{minipage}[t]{500 pt}
	\vspace{5mm}
	\emph{\textbf{Resumen}}
	\\[-2mm]
	\line(1,0){500}
	\\[-4.25 mm]
	\line(1,0){500}
	\vspace{\baselineskip}
}
{
	\normalsize
	\\[2mm]
	\footnotesize\textbf{Palabras clave: \footnotesize\@palabras}
	\\[-2mm]
	\line(1,0){500}
	\\[0.5cm]
	\end{minipage}
	\end{center}
}

% -------------------- Para las palabras clave -------------- %
\def\palabras#1{\gdef\@palabras{#1}}

%%%%%%%%%%%%%%%%%%%%%%%%%%%%%%%%%%%%

\lhead{B5602 - Entregable N° 3}
\chead{}
\rhead{Simulación DEP}	% Aquí va el numero de experimento, al igual que en el titulo
\lfoot{Ingeniería Electrónica}
\cfoot{\thepage\ de \pageref{LastPage}}
\rfoot{Universidad Nacional de Río Negro}

%%%%%%%%%%%%%%%%% PALABRAS CLAVE 
\palabras{Correlación, Variable Aleatoria, Gaussiana, Covarianza}
%%%%%%%%%%%%%%%%%
% Se escriben después del resumen y sintetizan los conceptos fundamentales del experimento a modo de etiquetas


%%%%%%%%%%%%%%%%
\begin{document}	% Inicio del documento
%%%%%%%%%%%%%%%%
\pdfbookmark[1]{Portada}{portada} 	% Marcador para el título

\begin{titlepage}
\centering
{\includegraphics[width=0.2\textwidth]{imagenes/LOGOUNRN.jpg}\par}
\vspace{0.5cm}
{\bfseries\large Universidad Nacional de Río Negro \par}
\vspace{0.5cm}
{\scshape\large Escuela de Producción, Tecnología y Medio Ambiente \par}
\vspace{0.5cm}
{\scshape\large Ingeniería Electrónica \par}
\vspace{3cm}
{\bfseries\Large Entregable N° 4 \par}
{\Large Detección de múltiples Señales inmersas en ruido\par}
\vfill
{\large \textbf{Alumno:} Mirko Manuel Pojmaevich\par}
{\large \textbf{Profesores:} Areta Javier, Marinsek Gonzalo\par}
{\large \textbf{Materia:} Procesos Estocásticos | \textbf{Código:} B5602\par}
\vspace{3cm}
{\large Fecha de entrega: 27 de octubre de 2022 \par}
\vspace{1cm}
\begin{table*}[!ht]
\begin{center}
\begin{tabular}{| c | c | c | c |}
\hline
\textbf{Rev.} & \textbf{Fecha} & \textbf{Profesor} & \textbf{Nota} \\ 
\hline
 &  & & \\
 \hline
 & & &  \\
\hline
\end{tabular}
\end{center}
\end{table*}
\end{titlepage}
%\maketitle							% Título
\newpage
\tableofcontents
\newpage
%\listoffigures
%\listoftables

\newpage
%%%%%%%%%%%%%%%%%%%%%%%
\section{Ejercicios}
\begin{mdframed}
	Considere la estimación de la DEP de una señal a partir de sus $m$ muestras, utilizando el periodograma dado por
	\begin{equation*}
		S_x(e^{j2\pi f})=\frac{1}{m}|X(e^{j2\pi f})|^2
	\end{equation*}
	donde $X(e^j2\pi f)$ es la TFD de la secuencia $x[n]$
	\begin{equation*}
		X(e^{j2\pi f})=\sum_{k=0}^{m-1}x[k]e^{j2\pi f}
	\end{equation*}
\begin{enumerate}
	\item Considere el periodograma de una señal iid de 128 muestras.
		Calcule analíticamente la forma de la DEP
		del proceso. Analice el efecto de la longitud de la secuencia.
	\item Genere 4 realizaciones de una secuencia iid Gaussiana de media cero y varianza unitaria de 128 muestras.
		Calcule y grafique el periodograma, y en la misma gráfica superponga la DEP analítica.
	\item Para hacer un análisis estadístico del periodograma, calcule $N = 2, 5, 10, 50, 100$ y $500$ periodogramas como
		en el punto anterior. En figuras separadas, grafique el promedio de los $N$ periodogramas, y su desviación
		estándar (para cada $f_k$), comparando con la DEP teórica.
	\item Analice los resultados obtenidos y comente sobre la precisión del periodograma para la estimación de la
		DEP de un proceso.
\end{enumerate}
\end{mdframed}
%
\newpage
\section{Ejercicios 1, 2 y 3}
%
Tras definir los datos, las secuencias y los vectores a transmitir, se los plotea juntos,
como se muestra en la figura \ref{fig:datos}.
Las señales fueron renombradas, a los datos X1 y X2 se los llama d1 y d2 respectivamente.
Las secuencias conservan el nombre, pero los mensajes T1 y T2, se los llama M1 y M2.

\begin{figure}[!ht]
\centering
\includegraphics[width=18cm]{imagenes/Señales.png}
\caption{Datos a transmitir, secuencia de codificación y mensajes a codificados}
\label{fig:datos}
\end{figure}

\newpage
\section{Ejercicio 4}

La primera señal recibida, con ruido aleatorio de magnitud menor a 0,1, se muestra en
la figura \ref{fig:transimcion01}

\begin{figure}[!ht]
\centering
\includegraphics[width=18cm]{imagenes/Transmicion_0.1.png}
\caption{Transmisión con ruido aleatorio de magnitud menor a 0,1}
\label{fig:transimcion01}
\end{figure}

\section{Ejercicio 5}

Las simulaciones fueron realizadas en Julia, lenguaje de programación que cuenta con 
una función filter de la que se desconocía al momento de simular, por lo que se realizó una convolución.
Para que la convolución actuara como el calculo de la correlación se invirtió el orden
de los elementos en el vector de la secuencia y se utilizó este vector invertido como
respuesta al impulso de filtro teórico. En la ecuación \ref{eq:corr} se muestra porque esta 
solución resulta en la correlación.

\begin{equation}
	\label{eq:corr}
	\begin{split}
		T\star h(\tau) :=& \int_{-\infty}^{\infty} T^*(t)\cdot h(t+\tau)dt\\
		T\ast h(\tau) :=& \int_{-\infty}^{\infty} T(t)\cdot h(t-\tau)dt\\
		T(t) =&\ T^*(t) \because T(t)\in \mathbb{R}\\
		\therefore 		T\ast h(-\tau) =&\ T\star h(\tau)
	\end{split}
\end{equation}

El resultado de este filtrado se puede apreciar en la figura \ref{fig:filtro0,1}

\begin{figure}[!ht]
\centering
\includegraphics[width=18cm]{imagenes/Filtrado_0.1.png}
\caption{Señales obtenidas tras calcular la correlación de la transmisión con cada código}
\label{fig:filtro0,1}
\end{figure}

En la figura \ref{fig:filtro0,1} se pueden apreciar 8 picos muy pronunciados en cada resultado.
Si se comparan los picos de cada señal filtrada, con los datos iniciales en la figura \ref{fig:datos},
se puede apreciar como el orden de estos picos coincide con los datos iniciales.

\begin{align*}
	d_{1} =
	\begin{bmatrix}
		{1,}\ 
		{1,}\ 
		{-1,}\ 
		{-1,}\ 
		{1,}\ 
		{1,}\ 
		{-1,}\ 
		{-1,}
	\end{bmatrix};\ \ \ 
	d_{2} =
	\begin{bmatrix}
		{-1,}\ 
		{1,}\ 
		{-1,}\ 
		{1,}\ 
		{1,}\ 
		{-1,}\ 
		{1,}\ 
		{-1,}
	\end{bmatrix}
\end{align*}

Esto nos da la pauta de que filtrando podemos recuperar el mensaje, estableciendo dos umbrales
y tratando los picos que sobrepasen el umbral superior como un 1 y los que caigan debajo del 
umbral inferior como -1.

\newpage
\section{Aumentar progresivamente el nivel de ruido}

\begin{figure}[!ht]
\centering
\includegraphics[width=18cm]{imagenes/Transmicion_0.5.png}
\includegraphics[width=18cm]{imagenes/Filtrado_0.5.png}
\caption{Transmisión y datos filtrados, con un ruido de magnitud menor a 0,5}
\label{fig:filtro05}
\end{figure}

\begin{figure}[!ht]
\centering
\includegraphics[width=18cm]{imagenes/Transmicion_1.0.png}
\includegraphics[width=18cm]{imagenes/Filtrado_1.0.png}
\caption{Transmisión y datos filtrados, con un ruido de magnitud menor a 1,0}
\label{fig:filtro10}
\end{figure}

\begin{figure}[!ht]
\centering
\includegraphics[width=18cm]{imagenes/Transmicion_1.5.png}
\includegraphics[width=18cm]{imagenes/Filtrado_1.5.png}
\caption{Transmisión y datos filtrados, con un ruido de magnitud menor a 1,5}
\label{fig:filtro15}
\end{figure}

\clearpage

\begin{figure}[!ht]
\centering
\includegraphics[width=18cm]{imagenes/Transmicion_2.0_primerIntento.png}
\includegraphics[width=18cm]{imagenes/Filtrado_2.0_primerIntento.png}
\caption{Transmisión y datos filtrados, con un ruido de magnitud menor a 2,0}
\label{fig:filtro201}
\end{figure}

En la figura \ref{fig:filtro201} se ve como algunos datos ya no muestran un pico tan pronunciado,
en relación a los picos de ruido. En este caso se ve como un pico negativo de ruido,
cercano al segundo dato de la señal roja, tiene una magnitud similar al cuarto dato de la misma señal.
Para corroborar si este fenómeno era consistente, o si esta situación desfavorable fue fruto del asar,
se repitió la simulación dos veces más.

\begin{figure}[!ht]
\centering
\includegraphics[width=18cm]{imagenes/Transmicion_2.0_segundoIntento.png}
\includegraphics[width=18cm]{imagenes/Filtrado_2.0_segundoIntento.png}
\caption{Transmisión y datos filtrados, con un ruido de magnitud menor a 2,0. Segundo intento}
\label{fig:filtro202}
\end{figure}

\begin{figure}[!ht]
\centering
\includegraphics[width=18cm]{imagenes/Transmicion_2.0_tercerIntento.png}
\includegraphics[width=18cm]{imagenes/Filtrado_2.0_tercerIntento.png}
\caption{Transmisión y datos filtrados, con un ruido de magnitud menor a 2,0. Tercer intento}
\label{fig:filtro203}
\end{figure}

Tras repetir el experimento dos veces más, se obtuvieron las figureas \ref{fig:filtro202}
y \ref{fig:filtro203}. La figura \ref{fig:filtro203} es alentadora, porque en ella existe umbral
claro mediante el cual se podría reconstruir la señal. Ningún pico de ruido supera una amplitud de 200,
el mas cercano ronda los 180 cerca del tercer dato de la señal azul, mientras que ningún dato cae por
debajo de este umbral, siendo el más bajo el primer dato de la señal azul, el cual ronda los 220.
Sin embargo, en la figura \ref{fig:filtro202}, se ve un pico de ruido cercano al segundo dato de la
señal azul que supera en magnitud al primer dato de la misma señal. Esto podría intentar solucionarse
estableciendo una ventana de tiempo entre picos y tomando el pico de mayor magnitud dentro de cada ventana.
Aun así, formalizar este mecanismo excede el objetivo de este trabajo, por lo que se concluirá que,
utilizando el método de los umbrales, un ruido con el doble de magnitud que los datos es el máximo que este
sistema tolera.

\begin{equation}
	S/N < 0,5
\end{equation}

Por curiosidad sobre como se deteriora la señal al seguir aumentando el nivel de ruido se repitió
la simulación con una magnitud de ruido menor a 2,5.

\begin{figure}[!ht]
\centering
\includegraphics[width=18cm]{imagenes/Transmicion_2.5.png}
\includegraphics[width=18cm]{imagenes/Filtrado_2.5.png}
\caption{Transmisión y datos filtrados, con un ruido de magnitud menor a 2,5}
\label{fig:filtro25}
\end{figure}

Resulta muy llamativo como la mayoría de los picos pueden seguir siendo claramente diferenciados, 
el problema surge cuando se compara el pico de ruido máximo con el pico de dato mínimo. 

\clearpage
\section{Optativos}
\subsection{Aumentar L}

Cambiando $L=100$ por $L=1000$, obtenemos la figura \ref{fig:L1000}.

\begin{figure}[!ht]
\centering
\includegraphics[height=6.5cm]{imagenes/Señales_L1000.png}
\includegraphics[height=6.5cm]{imagenes/Transmicion_2.5_L1000.png}
\includegraphics[height=6.5cm]{imagenes/Filtrado_2.5_L1000.png}
\caption{Señales, transmisión y datos filtrados, con un ruido de magnitud menor a 2,5. Con L=1000}
\label{fig:L1000}
\end{figure}

En este caso los picos de los datos se amplifican a tal punto que resulta posible aumentar el nivel de ruido
significativamente. En la figura \ref{fig:Ruido10} se muestra esto mismo. Pudiendo identificar los datos 
utilizando umbrales con una $S/N = 0,1$.

\begin{figure}[!ht]
\centering
\includegraphics[width=18cm]{imagenes/Transmicion_10_L1000.png}
\includegraphics[width=18cm]{imagenes/Filtrado_10_L1000.png}
\caption{Transmisión y datos filtrados, con un ruido de magnitud menor a 10}
\label{fig:Ruido10}
\end{figure}

\subsection{Calculo teórico del filtro}

\begin{equation}
	\begin{split}
		M[n] &:= 
		\sum_{
			i=0
		}^{
			N-1
		}
		d_i\cdot S[t-i\cdot L]\\
		T[n] &:= M[n] + N[n]\\
		T\star S[k] &= M\star S[k] + N\star S[k]\\
		N\star S[k] &= 0\\
		M\star S[k] &= \sum_{
			i=0
		}^{
			N-1
		}
		d_i\cdot L\cdot\delta[n-\frac{L}{2}-i*L]\\
		M\star S[k] &= 
		\begin{cases}
			L & \text{si, }k=\frac{L\cdot i}{2}\wedge d_i=1\\
			-L & \text{si, }k=\frac{L\cdot i}{2}\wedge d_i=-1\\
			0 & \text{c.c.}
		\end{cases}
		;\ i\in\mathbb{Z}
	\end{split}
\end{equation}

Esto no coincide con lo visto en las simulaciones, ver ruido en lugar de 0 entre los datos es esperable
ya que el calculo nos da el valor esperado, pero lo que encontramos es una realización. Lo que no es
esperado el la magnitud de los picos, en ocasiones llegando a $4\cdot L$. Se asume que se debe a algún
artefacto de la función de convolución de Julia y se lo tendrá en cuenta para trabajos futuros.

\subsection{Ruido no blanco}

Como las secuencias son generadas aleatoriamente y se esperaría que sean blancas, que el 
ruido no sea blanco no debería traer consecuencias. 
Conociendo las secuencias esto cambia, porque ahora podemos conocer el valor medio de cada secuencia,
el cual probablemente sea distinto de 0. Esto causa que en la practica esperemos ver, aproximadamente,
el ruido escalado por la media de secuencia. Por ejemplo, si nuestro ruido fuera un seno determinístico,
con un periodo significativamente mayor a L, esperaríamos ver este mismo seno escalado se propagarse a la
salida. Para probar esto se realizó una simulación con un ruido sinusoidal determinístico, que se muestra
en la figura \ref{fig:seno}.

\begin{figure}[!ht]
\centering
\includegraphics[width=18cm]{imagenes/RuidoSeno.png}
\caption{Ruido sinusoidal determinístico, con amplitud 10}
\label{fig:seno}
\end{figure}

Este ruido sinusoidal, produjo la transmisión que se muestra en la figura \ref{fig:TransSeno}.
La cual una vez filtrada resultó en la figura \ref{fig:FiltSeno}.

\begin{figure}[!ht]
\centering
\includegraphics[width=18cm]{imagenes/Transmicion_5_seno.png}
\caption{Transmisión con ruido sinusoidal determinístico, con amplitud 10}
\label{fig:TransSeno}
\end{figure}

\begin{figure}[!ht]
\centering
\includegraphics[width=18cm]{imagenes/Filtrado_5_Seno.png}
\caption{Señales filtradas de una transmisión con un ruido sinusoidal de amplitud 10}
\label{fig:FiltSeno}
\end{figure}

Se puede apreciar como los picos de los datos persisten claramente, a pesar del ruido
sinusoidal. El poder reconocerlos depende de si estos ocurren sobre un pico o 
un valle del seno. En la figura \ref{fig:FiltSeno}, pareciera que solo los picos positivos
se pueden recuperar, pero tras una inspección mas cuidadosa, como la que se muestra en la
figura \ref{fig:FiltSenoRect}, se puede ver como todos los picos resaltan sobre el seno.
La información de estas señales se podría recuperar con un notch que elimine este seno.
Un filtro pasa-altos no funcionaría porque al perder la componente de continua no se podría 
diferenciar un 1 de un -1. Esto filtrado se puede apreciar en la figura \ref{fig:senoFiltrado}.
Para obtener esta figura se utilizó un filtro elimina banda, butterworth de orden seis.
Las frecuencias con las que se parametrizó el filtro fueron 0,015 y 0,025, se llegó a ellas 
de la siguiente manera.

\begin{equation}
	\begin{split}
		\text{N° de muestras }&=800\\
		\text{N° de periodos }&=16\\
		f_{seno} &=
		\frac{ 16 }{ 800 }\\
		  &= 0,02\\
		f_{min}&=f_{seno}-0,005\\
		f_{max}&=f_{seno}+0,005\\
	\end{split}
\end{equation}

El valor 0,005 se eligió arbitrariamente.

\begin{figure}[!ht]
\centering
\includegraphics[width=18cm]{imagenes/Filtrado_5_Seno_detalle.png}
\caption{Detalle de los picos ocultos en los valles del ruido sinusoidal}
\label{fig:FiltSenoRect}
\end{figure}

\begin{figure}[!ht]
\centering
\includegraphics[width=18cm]{imagenes/SenoFiltrado.png}
\caption{Señales recuperadas tras filtrar el ruido sinusoidal con un notch}
\label{fig:senoFiltrado}
\end{figure}

\end{document}
